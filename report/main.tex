% This must be in the first 5 lines to tell arXiv to use pdfLaTeX, which is strongly recommended.
\pdfoutput=1
% In particular, the hyperref package requires pdfLaTeX in order to break URLs across lines.

\documentclass[11pt]{article}

% Remove the "review" option to generate the final version.
% \usepackage[review]{MAIN}
\usepackage[]{MAIN}

% Standard package includes
\usepackage{times}
\usepackage{latexsym}

% For proper rendering and hyphenation of words containing Latin characters (including in bib files)
\usepackage[T1]{fontenc}
% For Vietnamese characters
% \usepackage[T5]{fontenc}
% See https://www.latex-project.org/help/documentation/encguide.pdf for other character sets

% This assumes your files are encoded as UTF8
\usepackage[utf8]{inputenc}

% This is not strictly necessary, and may be commented out.
% However, it will improve the layout of the manuscript,
% and will typically save some space.
\usepackage{microtype}

% This is also not strictly necessary, and may be commented out.
% However, it will improve the aesthetics of text in
% the typewriter font.
\usepackage{inconsolata}


% If the title and author information does not fit in the area allocated, uncomment the following
%
%\setlength\titlebox{<dim>}
%
% and set <dim> to something 5cm or larger.

\title{Community Detection and Analysis on DBLP v9 Dataset: Centrality, Network Metrics, and Predictive Modeling}

% Author information can be set in various styles:
% For several authors from the same institution:
% \author{
%   Author 1 \and ... \and Author n \\
%   Address line \\ ... \\ Address line
% }
% if the names do not fit well on one line use
%         Author 1 \\ {\bf Author 2} \\ ... \\ {\bf Author n} \\
% For authors from different institutions:
\author{
  Xiang Zheng \\ 21307110169 \And  
  ... 
  \vspace{0.7cm} \\ % Add some space before the school name
  \hfill School of Data Science, Fudan University \hfill
  \And
  Author n \\ Address line
}

\begin{document}
\maketitle
\begin{abstract}
	abstract
\end{abstract}

\section{Introduction}

Network analysis has become an essential methodology for investigating complex systems across diverse domains. The \href{https://www.aminer.cn/citation}{DBLP dataset}, a comprehensive bibliographic repository of computer science publications, provides a rich foundation for exploring collaboration patterns and scholarly relationships. This project employs advanced graph analysis techniques to uncover the structural and functional properties of the DBLP v9 dataset.

The study is organized into five core components, each addressing a key aspect of network analysis:

\begin{itemize}
    \item \textbf{Preprocessing}: The raw DBLP dataset was transformed into a structured network by performing data cleaning, filtering, and splitting. Key preprocessing steps included handling missing data, defining nodes and edges, and attributing properties to network elements to prepare the dataset for analysis.
    \item \textbf{Community Detection}: Cohesive subgroups within the network were identified using algorithms such as \texttt{Louvain}, \texttt{Label Propagation}, and \texttt{Multi-level}. These communities reveal collaborative dynamics, research specializations, and the structural foundations of scholarly interactions.
    \item \textbf{Centrality Analysis}: Degree centrality and PageRank centrality were employed to identify influential nodes, revealing hubs and authoritative figures within the network. Additional metrics, including community diameters and average citations per author, were analyzed to characterize structural and functional network properties. The degree distribution was assessed to evaluate the network's topology, with visual representations provided in the visualization section.
    \item \textbf{Link Prediction}: Using the Cora-ML dataset—a citation network of machine learning research papers—supervised learning techniques were applied to predict potential future connections. Structural features such as common neighbors and Jaccard similarity were used, with model performance evaluated through metrics such as AUC and AP. The analysis provides insights into the evolution of citation patterns and network connectivity.
    \item \textbf{Visualization System}: An intuitive visualization system was developed to facilitate the interpretation of results. This system presents community structures, central nodes, and network statistics through graphical representations, offering a comprehensive overview of the network's topology and dynamics.
\end{itemize}

By integrating these analytical components, this study aims to elucidate the structural characteristics, key actors, and potential evolution of the DBLP co-authorship and publication network. The methodologies and findings presented contribute to a deeper understanding of scholarly collaboration patterns and academic network dynamics.



\section{Preprocessing}

\section{Community Mining}

\section{Centrality Measurement}

\section{Link Prediction}

\section{Visualization System}

\section{Conclusion}

\section*{Acknowledgements}


% % Entries for the entire Anthology, followed by custom entries
% \bibliography{anthology,custom}
% \bibliographystyle{acl_natbib}

% \appendix

% \section{Example Appendix}
% \label{sec:appendix}

% This is a section in the appendix.

\end{document}
